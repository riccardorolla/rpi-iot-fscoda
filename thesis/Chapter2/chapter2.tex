%*******************************************************************************
%****************************** Second Chapter *********************************
%*******************************************************************************
\nomenclature[z-Rpi1]{Rpi}{Raspberry Pi Version 1.0}
\nomenclature[z-Rpi2]{Rpi2}{Raspberry Pi Version 2.0}
\nomenclature[z-Rpi3]{Rpi3}{Raspberry Pi Version 3.0}
\nomenclature[z-CLR]{CLR}{Common Language Runtime}
\nomenclature[z-CL]{CL}{Common Language}
\nomenclature[z-GPU]{GPU}{Graphics Processing unit}
\nomenclature[z-CPU]{CPU}{Central Processing unit}
\nomenclature[z-SOC]{SOC}{System-On-Chip}
\nomenclature[z-GPIO]{GPIO}{General Purpose Input/Output}
\nomenclature[z-ML]{ML}{MetaLanguage, Functional Language}
\chapter{Preliminari}

\ifpdf
    \graphicspath{{Chapter2/Figs/Raster/}{Chapter2/Figs/PDF/}{Chapter2/Figs/}}
\else
    \graphicspath{{Chapter2/Figs/Vector/}{Chapter2/Figs/}}
\fi


\section{Raspberry Pi}
Il Raspberry Pi è un calcolatore di  dimensioni molto ridotte ed a basso costo realizzato dalla RaspberryPi Fondantion, esistono diversi versioni di questo piccolo computer  così come riportato dalla tabella.



La scelta per il progetto oggetto di questa tesi è stata per la versione 3  perchè mette a disposizione una potenza di calcolo superiorie a qualunche altro modello di RPi, grazie SoC BCM2837 all'interno del quale c'è  un 1Gb memoria, una CPU un quad-core ARM Cortex A53 a 64 bit a 1.2 Ghz ed una GPU VideoCore IV a 400Mhz, inoltre la Rpi3 è dotata del chip BCM43438 che fornisce connessione Wifi b/g/n e Bluetooth eliminando la necessità di utilizzare  un adatattore Wifi USB come avveniva nelle versioni precedenti.
L'antenna usata per le comunicazioni wireless si  trova sul bordo esterno della scheda, fuori dalla portata di possibili interferenze causate  da eventuali dispositivi e componenti aggiuntivi. L'integrazione delle comunicazioni wireless è una caratteristica importante se si vuole valutare l'aspetto di risparmio economico per una piattaforma IoT.

La Rpi ha quattro porte USB e una porta Ethernet 10/100 che condividono lo stesso bus gestito dal chip LAN9514. Questa soluzione può essere criticata  perchè non permette di sfruttare il 1Gbit sulla porta Ethernet, ma ai fini del nostro progetto non è un difetto che penalizza.


Nel dispositivo non è presente nessun tipo di memoria di massa, ma è presente un lettore di memorie MicroSD.
L'avvio del sistema avviene dalla scheda MicroSD all'interno della quale c'è una partizione con sopra il firmware, kernel e la configurazione del sistema.
Questo calcolatore è stato progettato per funzionare su diversi sistemi operativi Linux, RiscOS e Windows Core Iot.
Nell'appendice A descrivo come installare la versione Linux Raspian, il sistema operativo scelto per questo progetto.

\section{GPIO}
Tutte le versioni di questo computer hanno la possabilità di poter utilizzare GPIO  le GPIO sono delle porte che è possibile interagirgi

Nell'appendice B descrivo un semplice progetto che utilizza le porte GPIO


 

\section{Mono}
Mono è un progetto opensource coordinato da Xamarin, dal 2015 una sussidiaria di Microsoft, che ha lo scopo di realizzare  un insieme di strumenti compatibili con il Framework .NET di Microsoft ed aderenti allo standard Ecma anche per ambienti non Windows.
Il progetto mono implementa sia macchine virtuali chiamata Common Language Runtime per l'esecuzione dei programmi .NET per diverse piattaforme che compilatori per diversi linguaggi di programmazione.

Tra le CLR implementate c'è anche quella per Linux per processori x64 ARM e tra i compilatori di linguaggi implemetanti da Mono c'è sia il compilatore per \texttt{C\#} che quello per \texttt{F\#} che verrano utilizzati in questo progetto.

Nell'appendice C sarà  descritto come installare Mono e il compilatore \texttt{F\#} su Rpi3.


\section{\texttt{ML}_{CoDa}}
 \texttt{ML}_{CoDa} è un linguaggio di programmazione orientanto al Contesto. Questo linguaggio consente con alcuni costrutti di adattare il programma al contesto in cui viene eseguito. Il prototipo del linguaggio \texttt{ML}_{CoDa} che userò è un estensione del linguaggio funzionale F#, della famiglia dei linguaggi ML.
 
 
 Nell'appendice D descrivo come testare gli esempi di \texttt{ML}_{CoDa} su Mono installato su Rpi3
 Nell'appendice E implemento il progetto button-led in \texttt{ML}_{CoDa} e i
\section{Node.js}

\section{Express.js}

\section{Telegram.js}

\section{Microsoft Vision cloud...} 

